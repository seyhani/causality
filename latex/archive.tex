
One use of event structure is to give denotational semantics of the language of parallel
processes that reflect the parallelism in processes as causal independence between events.
The nature of the events, how they interact with the environment,
is specified in the language by associating each event with a label from the synchronization
algebra $L$.
The language we shall use is one where processes communicate by events of synchronization
with no value passing.
Its syntax has the form:
\begin{align*}
    p ::= nil | \alpha p | p_0 + p_1 | p_0 \times p_1 | p\lceil \Lambda | p[\Xi] | x | recx.p
\end{align*}
where $x$ is in some set of variables $X$ over processes, $\alpha$ is a label,
$\Lambda$ is a subset of labels, in $p[\Xi]$ the symbol $\Xi$ denotes a relabelling function between
two sets of labels.

Informally, the product $p_0 \times p_1$ is a form of parallel composition which introduces
arbitrary events of synchronization between processes.
Unwanted synchronizations can be restricted away with the help of the restriction operation
$p\lceil \Lambda$ and the existing events renamed with the relabelling operation $p[\Xi]$.
So in this way, we can define specialized parallel compositions of the kind that appear in
CCS and CSP, for example.
To explain formally the behavior of the constructs in the language we describe them as
constructions on labeled event structures, so a closed process term in this language is to
denote a \textbf{stable event structure} but where the events are labeled.

\subsection{Denotational Semantics}

\begin{definition}
    Define an environment for process variables to be a function $\rho$
    from process variables $X$ to labeled event structures.
    For a term $t$ and environment $\rho$, define the denotation of $t$ with
    respect to $\rho$ written $\llbracket t \rrbracket \rho$ by the following
    structural induction syntactic operators appear on the left and their
    semantics counterparts on the right.
    \begin{equation*}
        \begin{aligned}[c]
            \sem{nil}\rho       & = (\emptyset,\emptyset)         \\
            \sem{x}\rho         & = \rho(x)                       \\
            \sem{\alpha t}\rho  & = \alpha(\sem{t}\rho)           \\
            \sem{t_1 + t_2}\rho & = \sem{t_1}\rho + \sem{t_2}\rho \\
        \end{aligned}
        \qquad
        \begin{aligned}[c]
            \sem{t\lceil \Lambda}\rho & = \sem{t}\rho \lceil \Lambda         \\
            \sem{t[\Xi]}\rho          & = \sem{t}\rho[\Xi]                   \\
            \sem{t_1 \times t_2}\rho  & = \sem{t_1}\rho \times \sem{t_2}\rho \\
            \sem{recx.t}\rho          & = fix\Gamma                          \\
        \end{aligned}
    \end{equation*}
    where $\Gamma$ is an operation on labelled event structures given by
    $\Gamma(E) = \sem{t}\rho[E / x]$ and $fix$ is the least-fixed-point operator.
\end{definition}