\section{Semantics of Communicating Processes}
\subsection{Normal DyNetKAT}
We define the Normal DyNetKAT grammar as follows:
\begin{align*}
    F ::= & \alpha\cdot\pi                                                                           \\
    D ::= & \bot ~|~ F;D ~|~ x?F;D ~|~ x!F;D ~|~ D \parallel D ~|~ D \oplus D ~|~ \delta_{\mc{L}}(D) \\
          & \mc{L} = \s{c ~|~ c ::= x?F ~|~ x!F}
\end{align*}

Lemma 9 in \cite{dynetkat} says that for a given guarded DyNetKAT term
$p$ there exists a DyNetKAT term in normal form $q$ for which we have:
\begin{equation*}
    E_{DNK} \vdash p \equiv q
\end{equation*}
where $E_{DNK}$ is DyNetKAT axiom system.

In the following we describe how to compose event structures and
then labeled events structures.
Using these definitions we provide a semantic for Normal DyNetKAT
terms.

\subsection{Empty event structure}
We use $(\e,\e)$ to denote an empty labeled event structure with
an empty set of events and an empty set of labels.

\subsection{Prefix}

\begin{definition}
    Let $(\mathcal{E},L,l)$ be a labelled event structure.
    Let $\alpha$ be a label.
    Define $\alpha(\mathcal{E},L,l)$ to be a labelled event structure $(\alpha \mathcal{E},L',l')$
    with labels:
    \begin{align*}
        L' = \s{\alpha} \cup L
    \end{align*}
    and
    $$
        l'(e') = \begin{cases}
            \alpha & \text{if } e' = (0,\alpha) \\
            l(e)   & \text{if } e' = (1,e)
        \end{cases}
    $$
    for all $e' \in \mathcal{E'}$.
\end{definition}
The configurations of $\alpha E$, a prefixed labeled event structure,
have the simple and expected characterization.
(By $\mathcal{F}(E)$ of a labeled event structure $E$ we shall understand the set
of configurations of the underlying event structure)

\subsection{Sum}
\begin{definition}
    Let $\mr{E_0} = (E_0,\#_0,\vdash_0,L_0,l_0)$ and
    $\mr{E_1} = (E_1,\#_1,\vdash_1,L_1,l_1)$ be labelled event structures.
    Their sum $\mr{E_0} + \mr{E_1}$, is defined to be the structure $(E,\#,\vdash,l)$
    with events $E = \s{(0,e)|e \in E_0} \cup \s{(1,e)|e \in E_1}$,
    the disjoint union of sets $E_0$ and $E_1$,
    with injections $\iota_k: E_k \rightarrow E$, given by
    $\iota_k(e) = (k,e)$, for $k=0,1$, conflict relation
    \begin{align*}
        e \# e' \iff & \exists e_0,e_0'. e = \iota_0(e_0)
        \wedge e' = \iota_0(e_0') \wedge e_0 \#_0e_0'                       \\
                     & \text{or } \exists e_1,e_1'. e = \iota_1(e_1) \wedge
        e' = \iota_1(e_1') \wedge e_1 \#_1 e_1'                             \\
                     & \text{or } \exists e_0,e_1.(e=\iota_1(e_0)
        \wedge e' =\iota_1(e_1)) \text{ or }
        (e'=\iota_1(e_0) \wedge e =\iota_1(e_1))
    \end{align*}
    and enabling relation
    \begin{align*}
        X \vdash e \iff & X \in Con \wedge e \in E \wedge                   & \\
                        & (\exists X_0 \in Con_0,e_0 \in E_0.X = \iota_0X_0
        \wedge e = \iota_0(e_0) \wedge X_0 \vdash_0 e_0) \text{ or }          \\
                        & (\exists X_1 \in Con_1,e_1 \in E_1.X = \iota_1X_1
        \wedge e = \iota_1(e_1) \wedge X_1 \vdash_1 e_1)                      \\
    \end{align*}
    We define the set of labels as $L_0 \cup L_1$ and the labelling function as:
    $$
        l(e) = \begin{cases}
            l_0(e_0) & \text{ if } e = \iota_0(e_0) \\
            l_1(e_1) & \text{ if } e = \iota_1(e_1)
        \end{cases}
    $$
\end{definition}

\subsection{Product}
In the product of two event structures, their events of synchronization are those pairs of events $(e_0,e_1)$, one from each event structure;
if $e_0$ is labelled $\alpha_0$ and $e_1$ is labelled $\alpha_1$ the synchronization event is
then labelled $(\alpha_0,\alpha_1)$.
Events need not synchronize however; an event in one component may not synchronize with
any event in the other.
We shall use events of the form $(e_0,*)$ to stand for the occurrence of an event $e_0$
from one component unsynchronized with any event of the other.
Such an event will be labeled by $(\alpha_0,*)$ where $\alpha_0$ is the original label of $e_0$
and * is a sort of undefined.

\begin{definition}
    Let $\mr{E_0} = (E_0,\#_0,\vdash_0,L_0,l_0)$ and
    $\mr{E_1} = (E_1,\#_1,\vdash_1,L_1,l_1)$
    be labeled event structures.
    Define their product $\mr{E_0} \times \mr{E_1}$ to be the structure $\mr{E} = (E,\#,\vdash,L,l)$
    consisting of events $E$ of the form
    \begin{align*}
        E_0 \times_* E_1 =
        \s{(e_0,*)|e_0 \in E_0}
        \cup \s{(*,e_1)|e_1 \in E_1}
        \cup \s{(e_0,e_1)| e_0 \in E_0 \wedge e_1 \in E_1}
    \end{align*}
    with projections $\pi_i : E \rightarrow_* E_i$,
    given by $\pi_i(e_0,e_1) = e_i$, for $i=0,1$, reflexive conflict relation $\doublevee$ given by
    \begin{align*}
        e \doublevee e' \iff \pi_0(e) \doublevee_0 \pi_0(e') \text{ or }
        \pi_1(e) \doublevee_1 \pi_1(e')
    \end{align*}
    for all $e,e'$ we use $Con$ for the conflict-free finite sets,
    enabling relation $\vdash$ given by
    \begin{align*}
         & X \vdash e \iff X \in Con \wedge e \in \mathcal{E} \wedge        \\
         & (\pi_0(e)\text{ is defined } \Rightarrow \pi_0X\vdash_0\pi_0(e))
        \wedge (\pi_1(e)\text{ is defined } \Rightarrow \pi_1X\vdash_1\pi_1(e))
    \end{align*}
    Its set of labels is
    \begin{align*}
        L_0 \times_* L_1 = \s{ (\alpha_0,*)|\alpha_0 \in L_0}
        \cup \s{(*,\alpha_1)|\alpha_1 \in L_1}
        \cup \s{(\alpha_0,\alpha_1)|\alpha_0 \in L_0 \wedge \alpha_1 \in L_1}
    \end{align*}
    with projections: $\lambda_i: E \rightarrow_* E_i$ given by
    $\lambda_i(\alpha_0,\alpha_1) = \alpha_i$, for $i=0,1$.
    Its labeling function is defined to act on an event $e$ so
    \begin{align*}
        l(e) = (l_0\pi_0(e),l_1\pi_1(e))
    \end{align*}
\end{definition}

\subsection{Restriction}

\begin{definition}

    Let $\mr{E} = (E,\#,\vdash,L,l)$ be a labelled event structure.
    Let $\Lambda$ be a subset of labels.
    Define the restriction $\mr{E}\lceil \Lambda$ to be $(E',\#',\vdash',L\cap \Lambda,l')$
    where $(E',\#',\vdash')$ is the restriction of $(E,\#,\vdash)$
    to events $\s{e \in E|l(e) \in \Lambda}$ and the labeling function $l'$
    is the restriction of the original labeling function to the domain $L \cap \Lambda$.
\end{definition}

\subsection{Denotational Semantics}

\begin{definition}
    Let $\mc{A}$ be an alphabet of letters of the form
    $\alpha \cdot \pi$,
    $x?F$, and $x!F$.
    We define the semantic map of Normal DyNetKAT terms
    $\sem{ \ }: D \ra \mathbb{E}$ where
    $\mathbb{E}$ is the set of all event structures with
    labels in $\mc{A}$ as follows:
    \begin{align*}
        \sem{\bot}      & = (\emptyset,\emptyset)                  \\
        \sem{\alpha; t} & = \alpha(\sem{t})                        \\
        \sem{t_1 \oplus t_2}
                        & = \sem{t_1} + \sem{t_2}                  \\
        \sem{\delta_{\mc{L}}(t)}
                        & = \sem{t} \lceil \mc{A} \setminus \mc{L} \\
        \sem{t_1 \parallel t_2}
                        & = \sem{t_1} \times \sem{t_2}
    \end{align*}
    Where $\mc{L}$ is a subset of $\mc{A}$.
\end{definition}

% \subsection{Examples}
% Consider the following DyNetKAT program:
% \begin{equation*}
%     SDN = a \ra b; a \ra c
% \end{equation*}
% It encodes a network where a packet on $a$ is first forwarded to $b$
% and the next packet is forwarded to $c$.
% Let $\mr{E} = (E,\#,\vdash)$ be the event structure of $SDN$ so we have:
% \begin{align*}
%     E & = \s{ab,ac}                     \\
%     \# = \e                             \\
%       & \e \vdash ab, \s{ab} \vdash{ac} \\
%     L & = \s{a\ra b, a\ra c}            \\
%       & l(ab) = a\ra b, l(ac) = a \ra c
% \end{align*}
% Figure \ref{fig:ex:sem:prefix} shows the configurations of $\mr{E}$.


% \begin{figure}
%     \centering
%     \begin{tikzpicture}
%         \crd[left]{0}{0}{$\emptyset$}
%         \crd[left]{0}{1}{$\s{ab}$}
%         \crd[left]{0}{2}{$\s{ab,ac}$}
%         \draw [ultra thick] (0,0) -- (0,1);
%         \draw [ultra thick] (0,1) -- (0,2);
%     \end{tikzpicture}
%     \caption{}
%     \label{fig:ex:sem:prefix}
% \end{figure}