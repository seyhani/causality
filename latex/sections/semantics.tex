One use of event structure is to give denotational semantics of the language of parallel
processes that reflect the parallelism in processes as causal independence between events.
The nature of the events, how they interact with the environment,
is specified in the language by associating each event with a label from the synchronization
algebra $L$.
The language we shall use is one where processes communicate by events of synchronization
with no value passing.
Its syntax has the form:
\begin{align*}
    p ::= nil | \alpha p | p_0 + p_1 | p_0 \times p_1 | p\lceil \Lambda | p[\Xi] | x | recx.p
\end{align*}
where $x$ is in some set of variables $X$ over processes, $\alpha$ is a label,
$\Lambda$ is a subset of labels, in $p[\Xi]$ the symbol $\Xi$ denotes a relabelling function between
two sets of labels.

Informally, the product $p_0 \times p_1$ is a form of parallel composition which introduces
arbitrary events of synchronization between processes.
Unwanted synchronizations can be restricted away with the help of the restriction operation
$p\lceil \Lambda$ and the existing events renamed with the relabelling operation $p[\Xi]$.
So in this way, we can define specialized parallel compositions of the kind that appear in
CCS and CSP, for example.
To explain formally the behavior of the constructs in the language we describe them as
constructions on labeled event structures, so a closed process term in this language is to
denote a \textbf{stable event structure} but where the events are labeled.

\subsection{Nil}
The term $nil$ represents the $nil$ process that has stopped and refuses to perform any event;
it will denoted by the empty labelled event structure $(\emptyset,\emptyset,\emptyset,\emptyset,\emptyset)$
no events, no labels.

\subsection{Prefix}

\begin{definition}
    Let $(\mathcal{E},L,l)$ be a labelled event structure.
    Let $\alpha$ be a label.
    Define $\alpha(\mathcal{E},L,l)$ to be a labelled event structure $(\alpha \mathcal{E},L',l')$
    with labels:
    \begin{align*}
        L' = \s{\alpha} \cup L
    \end{align*}
    and
    $$
        l'(e') = \begin{cases}
            \alpha & \text{if } e' = (0,\alpha) \\
            l(e)   & \text{if } e' = (1,e)
        \end{cases}
    $$
    for all $e' \in \mathcal{E'}$.
\end{definition}
The configurations of $\alpha E$, a prefixed labeled event structure,
have the simple and expected characterization.
(By $\mathcal{F}(E)$ of a labeled event structure $E$ we shall understand the set
of configurations of the underlying event structure)

\begin{proposition}

    Let $E$ be a labelled event structure. Let $\alpha$ be a label.
    \begin{align*}
        x \in \mathcal{F}(\alpha E) \iff x = \emptyset \text{ or }
        [(0,\alpha)\in x \amp \s{e | (1,e)\in x} \in \mathcal{F}(E)]
    \end{align*}

\end{proposition}

\subsection{Sum}
A sum $p_0 + p_1$ behaves like $p_0$ or $p_1$; which branch of a sum is followed will
often be determined by the context and what kinds of events the process is restricted to.

\begin{definition}
    Let $E_0 = (\mathcal{E}_0,\#_0,\vdash_0,L_0,l_0)$ and
    $E_1 = (\mathcal{E}_1,\#_1,\vdash_1,L_1,l_1)$ be labelled event structures.
    Their sum $E_0 + E_1$, is defined to be the structure $(\mathcal{E},\#,\vdash,l)$
    with events $\mathcal{E} = \s{(0,e)|e \in \mathcal{E}_0} \cup \s{(1,e)|e \in \mathcal{E}_1}$,
    the disjoint union of sets $\mathcal{E}_0$ and $\mathcal{E}_1$,
    with injections $\iota_k: \mathcal{E}_k \rightarrow \mathcal{E}$, given by
    $\iota_k(e) = (k,e)$, for $k=0,1$, conflict relation
    \begin{align*}
        e \# e' \iff & \exists e_0,e_0'. e = \iota_0(e_0)
        \amp e' = \iota_0(e_0') \amp e_0 \#_0e_0'                         \\
                     & \text{or } \exists e_1,e_1'. e = \iota_1(e_1) \amp
        e' = \iota_1(e_1') \amp e_1 \#_1 e_1'                             \\
                     & \text{or } \exists e_0,e_1.(e=\iota_1(e_0)
        \amp e' =\iota_1(e_1)) \text{ or }
        (e'=\iota_1(e_0) \amp e =\iota_1(e_1))
    \end{align*}
    and enabling relation
    \begin{align*}
        X \vdash e \iff & X \in Con \amp e \in \mathcal{E} \amp                       & \\
                        & (\exists X_0 \in Con_0,e_0 \in \mathcal{E}_0.X = \iota_0X_0
        \amp e = \iota_0(e_0) \amp X_0 \vdash_0 e_0) \text{ or }                        \\
                        & (\exists X_1 \in Con_1,e_1 \in \mathcal{E}_1.X = \iota_1X_1
        \amp e = \iota_1(e_1) \amp X_1 \vdash_1 e_1)                                    \\
    \end{align*}
    We define the set of labels as $L_0 \cup L_1$ and the labelling function as:
    $$
        l(e) = \begin{cases}
            l_0(e_0) & \text{ if } e = \iota_0(e_0) \\
            l_1(e_1) & \text{ if } e = \iota_1(e_1)
        \end{cases}
    $$
\end{definition}
The configurations of a sum are obtained from copies of the configurations of the components
identified at their empty configurations.

\begin{proposition}

    Let $E_0$ and $E_1$ be labelled event structures.
    \begin{align*}
        x \in \mathcal{F}(E_0+E_1) \iff (\exists x_0 \in \mathcal{F}(E_0).x=\iota_0x_0)
        \text{ or } (\exists x_1 \in \mathcal{F}(E_1).x=\iota_1x_1)
    \end{align*}

\end{proposition}

\subsection{Product}

A product process $p_0 \times p_1$ behaves like $p_0$ and $p_1$ set in parallel.
Their events of synchronization are those pairs of events $(e_0,e_1)$, one from each process;
if $e_0$ is labelled $\alpha_0$ and $e_1$ is labelled $\alpha_1$ the synchronization event is
then labelled $(\alpha_0,\alpha_1)$.
Events need not synchronize however; an event in one component may not synchronize with
any event in the other.
We shall use events of the form $(e_0,*)$ to stand for the occurrence of an event $e_0$
from one component unsynchronized with any event of the other.
Such an event will be labeled by $(\alpha_0,*)$ where $\alpha_0$ is the original label of $e_0$
and * is a sort of undefined.

In fact we shall often want to take the first or second coordinates of such paris and,
of course, this could give the value * which we think of as undefined,
so that, in effect, we are working with partial functions with * understood to be undefined.
We can keep expressions tidier by adopting some conventions about how to treat this undefined value
when it appears in expressions and assertions.
\begin{align*}
     & \Theta(e) \in X \Rightarrow \Theta(e) \text{is defined, and }                                      \\
     & \Theta(e) = \Theta(e') \Rightarrow \Theta(e) \text{is defined } \amp \Theta(e')\text{ is defined}.
\end{align*}
We adopt a similar strict interpretation for function application.
So if $f$ is a function applied to some value, denoted by $a$, the $f(a)$ is undefined (gives *)
if $a$ is undefined.
As usual we represent the image of a set under a partial function by
\begin{align*}
    \Theta X = \s{\Theta(e)|e \in X \amp \Theta(e) \text{ is defined}}
\end{align*}

\begin{definition}

    Let $E_0 = (\mathcal{E}_0,\#_0,\vdash_0,L_0,l_0)$ and $E_1 = (\mathcal{E}_1,\#_1,\vdash_1,L_1,l_1)$
    be labelled event structures.
    Define their product $E_0 \times E_1$ to be the structure $E = (\mathcal{E},\#,\vdash,L,l)$
    consisting of events $\mathcal{E}$ of the form
    \begin{align*}
        \mathcal{E}_0 \times_* \mathcal{E}_1 =
        \s{(e_0,*)|e_0 \in \mathcal{E}_0}
        \cup \s{(*,e_1)|e_1 \in \mathcal{E}_1}
        \cup \s{(e_0,e_1)| e_0 \in \mathcal{E}_0 \amp e_1 \in \mathcal{E}_1}
    \end{align*}
    with projections $\pi_i : \mathcal{E} \rightarrow_* \mathcal{E}_i$,
    given by $\pi_i(e_0,e_1) = e_i$, for $i=0,1$, reflexive conflict relation $\doublevee$ given by
    \begin{align*}
        e \doublevee e' \iff \pi_0(e) \doublevee_0 \pi_0(e') \text{ or }
        \pi_1(e) \doublevee_1 \pi_1(e')
    \end{align*}
    for all $e,e'$ we use $Con$ for the conflict-free finite sets,
    enabling relation $\vdash$ given by
    \begin{align*}
         & X \vdash e \iff X \in Con \amp e \in \mathcal{E} \amp            \\
         & (\pi_0(e)\text{ is defined } \Rightarrow \pi_0X\vdash_0\pi_0(e))
        \amp (\pi_1(e)\text{ is defined } \Rightarrow \pi_1X\vdash_1\pi_1(e))
    \end{align*}
    Its set of labels is
    \begin{align*}
        L_0 \times_* L_1 = \s{ (\alpha_0,*)|\alpha_0 \in L_0}
        \cup \s{(*,\alpha_1)|\alpha_1 \in L_1}
        \cup \s{(\alpha_0,\alpha_1)|\alpha_0 \in L_0 \amp \alpha_1 \in L_1}
    \end{align*}
    with projections: $\lambda_i: \mathcal{E} \rightarrow_* \mathcal{E}_i$ given by
    $\lambda_i(\alpha_0,\alpha_1) = \alpha_i$, for $i=0,1$.
    Its labelling function is defined to act on an event $e$ so
    \begin{align*}
        l(e) = (l_0\pi_0(e),l_1\pi_1(e))
    \end{align*}
\end{definition}
We characterize the configurations of the product of two event structures in terms
of their configurations.

\begin{proposition}

    Let $E_0 \times E_1$ be the product of labelled event structures with projections
    $\pi_0,\pi_1$.
    Let $x \subseteq \mathcal{E}_0 \times \mathcal{E}_1$, the events of the product.
    Then $x \in \mathcal{F}(\mathcal{E}_0 \times \mathcal{E}_1)$ iff
    \begin{align*}
         & \pi_0(x)  \in \mathcal{F}(E_0) \amp \pi_1x \in \mathcal{F}(E_1)                                      \\
         & \forall e,e' \in x.\pi_0(e)=\pi_0(e') \text{ or } \pi_1(e) = \pi_1(e') \Rightarrow e = e'            \\
         & \forall e \in x \exists y \subseteq x. \pi_0 y \in \mathcal{F}(E_0) \amp \pi_1y \in \mathcal{F}(E_1)
        \amp e \in y \amp |y| < \text{infinite}                                                                 \\
         & \forall e,e' in x. e \neq e' \Rightarrow \exists y \subseteq x.\pi_0 y \in \mathcal{F}(E_0)
        \amp \pi_1y \in \mathcal{F}(E_1) \amp (e \in y \iff e' \not \in y)
    \end{align*}
\end{proposition}
The Proposition above expresses the intuition that an allowable behavior of the product
of two processes is precisely that which projects to allowable behaviors in the component
processes.
The complicated-looking conditions (c) and (d) are there just to ensure that the family
of sets is finitary and coincidence-free.

\subsection{Restriction}

The restriction $t \lceil \Lambda$ behaves like the process $p$ but with its events
restricted to those with labels that lie in the set $\Lambda$.

\begin{definition}

    Let $E = (\mathcal{E},\#,\vdash,L,l)$ be a labelled event structure.
    Let $\Lambda$ be a subset of labels.
    Define the restriction $E\lceil \Lambda$ to be $(\mathcal{E'},\#',\vdash',L\cap \Lambda,l')$
    where $(\mathcal{E'},\#',\vdash')$ is the restriction of $(\mathcal{E},\#,\vdash)$
    to events $\s{e \in \mathcal{E}|l(e) \in \Lambda}$ and the labelling function $l'$
    is the restriction of the original labelling function to the domain $L \cap \Lambda$.

\end{definition}

\begin{proposition}

    Let $E = (\mathcal{E},\#,\vdash,L,l)$ be a labelled event structure.
    Let $\Lambda \subseteq L$.
    \begin{align*}
        x \in \mathcal{F}(E\lceil\Lambda) \iff x \in \mathcal{F}(E) \amp e \in x.l(e) \in \Lambda
    \end{align*}

\end{proposition}

\subsection{Relabelling}

A relabelled process $p[\Xi]$ behaves like $p$ but with the events relabelled according to $\Xi$.

\begin{definition}

    Let $E = (\mathcal{E},\#,\vdash,L,l)$ be a labelled event structure.
    Let $\Lambda,L'$ be sets of labels and $\Xi: \Lambda \rightarrow L'$.
    Define the relabelling $E[\Xi]$ to be $(\mathcal{E},\#,\vdash,L',l')$ where
    $$
        l'(e) = \begin{cases}
            \Xi l(e) & \text{if } l(e) \in \Lambda \\
            l(e)     & \text{otherwise}
        \end{cases}
    $$

\end{definition}
\subsection{Denotational Semantics}

\begin{definition}
    Define an environment for process variables to be a function $\rho$
    from process variables $X$ to labeled event structures.
    For a term $t$ and environment $\rho$, define the denotation of $t$ with
    respect to $\rho$ written $\llbracket t \rrbracket \rho$ by the following
    structural induction syntactic operators appear on the left and their
    semantics counterparts on the right.
    \begin{equation*}
        \begin{aligned}[c]
            \sem{nil}\rho       & = (\emptyset,\emptyset)         \\
            \sem{x}\rho         & = \rho(x)                       \\
            \sem{\alpha t}\rho  & = \alpha(\sem{t}\rho)           \\
            \sem{t_1 + t_2}\rho & = \sem{t_1}\rho + \sem{t_2}\rho \\
        \end{aligned}
        \qquad
        \begin{aligned}[c]
            \sem{t\lceil \Lambda}\rho & = \sem{t}\rho \lceil \Lambda         \\
            \sem{t[\Xi]}\rho          & = \sem{t}\rho[\Xi]                   \\
            \sem{t_1 \times t_2}\rho  & = \sem{t_1}\rho \times \sem{t_2}\rho \\
            \sem{recx.t}\rho          & = fix\Gamma                          \\
        \end{aligned}
    \end{equation*}
    where $\Gamma$ is an operation on labelled event structures given by
    $\Gamma(E) = \sem{t}\rho[E / x]$ and $fix$ is the least-fixed-point operator.
\end{definition}
