\section{Causal Model of Unsafe Behavior in Event Structure}

\subsection{Causal Model of Event Structure}
Let $\mathrm{E} = (E,\#,\vdash)$ be an event structure where
$E = \s{e_1, e_2, ...,e_n}$ we define the causal model
$\mc{M} = (\mc{S},\mc{F},\mc{E})$ of unsafe behavior
in $\mr{E}$ where
$\mathcal{S} = (\mathcal{U},\mathcal{V},\mathcal{R})$.
We define $\mathcal{U}$ to be empty and $\mathcal{V}$
consisting of boolean variables as follows:
\begin{align*}
    \mathcal{V} = & \s{C_{e_i,e_j} ~|~  1 \leq i < j \leq n.
    e_i \in E \wedge e_j \in E}                                \\
                  & \cup \s{EN_{s,e} ~|~ s \in \mathcal{P}(E),
    e \in E. e \not \in s }                                  \\
                  & \cup \s{M_{s,e} ~|~ s \in \mathcal{P}(E),
        e \in E. e \not \in s } \cup \s{PV}
\end{align*}
For each variable $X \in \mathcal{V}$ we define $\vec V_X$ 
as a vector of all variables in $\mathcal{V} \setminus \s{X}$.
For $x,y \in \mathcal{P}(E)$ we say $x$ is covered by $y$ written $ x \prec y$ iff:
\begin{align*}
    x \subseteq y \wedge x \neq y \wedge
    (\forall z. x \subseteq z \subseteq y \Rightarrow x = z
    \vee y = z)
\end{align*}
We define the functions in $\mathcal{F}$ as follows:
$$
    \f{C_{e,e'}} = \begin{cases}
        true  & \text{ if } e \# e' \\
        false & \text{ otherwise }
    \end{cases}
$$
$$
    \f{M_{s,e}} = \begin{cases}
        Min(s,e) \wedge Con(s) & \text{ if } s \vdash_{min} e \\
        false                  & \text{ otherwise }
    \end{cases}
$$
\begin{align*}
    \f{EN_{s,e}} & =
    \left(
    M_{s,e} \bigvee
    \left(
    \bigvee_{s'\prec s}EN_{s',e}
    \right)
    \right)
    \bigwedge
    Con(s)
\end{align*}
Where we have:
\begin{align*}
    Con(s)   & =   \left(
    \bigwedge_{ 1\leq j<j' \leq n \wedge e_j,e_{j'} \in s}
    \neg C_{e_j,e_{j'}}
    \right)               \\
    Min(s,e) & = \left(
    \bigwedge_{s' \subseteq E. (s' \subset s \vee s \subset s')
        \wedge e \notin s'}
    \neg M_{s',e}
    \right)
\end{align*}
Let $\mathbb{E}$ be the all triples of the form
$(E,\#',\vdash')$ on the set of events $E$ where
$\#' \subseteq E \times E$ and
$\vdash' \subseteq \mc{P}(E) \times E$.
We define a function
$ES: \times_{V \in \mc{V}\setminus \s{PV}} \mc{R}(V) \rightarrow \mathbb{E}$ which intuitively returns the event structure
that can be derived from the values of the endogenous variables
in the model excluding $PV$.
Let $\vec v$ be a vector of the values of the variables
in $\mc{V} \setminus \s{PV}$ and $\vec v(V)$ be the value of
 $V$ in $\vec v$ for each $V \in \mc{V} \setminus \s{PV}$.
We define the function $ES$ so that if
$ES(\vec v) = (E,\#',\vdash')$ we have:
\begin{align*}
    \forall e,e' \in E. e \#' e' \wedge e' \#' e
     & \iff \vec{v}(C_{e,e'}) = \T \\
    \forall s \in \mathcal{P}(E), e \in E.  s \vdash' e
     & \iff \vec{v}(EN_{s,e}) = \T
\end{align*}
We define $\mathcal{E}$, the set of all allowable
settings of the endogenous variables, to be the vector of
values such as $\vec v'$ for which $ES(\vec v)$ is an
event structure.

Finally, we encode the property violation or unsafe behaviors
as a predicate on the set of configurations of the
event structure returned by $ES$.

\subsection{Actual Cause of Unsafe Behavior}

Using the definition of the actual cause in the extended causal 
model, we can rewrite the definition for unsafe behaviors in 
event structures as follows:

\begin{definition}
    Let $\mr{E} = (E,\#,\vdash)$ be an event structure and
    $\mc{M}$ be the causal model of an unsafe behavior in
    $\mr{E}$ where unsafe behavior is specified as the
    function $F_{PV}$ in $\mc{M}$.
    We say $\vec X = \vec x$ is an actual cause of 
    the unsafe behavior in $\mr{E}$ if the following
    conditions hold:
    \begin{itemize}
        \item  \textbf{AC1.} $M\models \vec X = \vec x
                  \wedge PV = \T$
        \item  \textbf{AC2. }There exists a partition $(\vec Z, \vec W)$ of $\mathcal{V}$ with $\vec X \subseteq \vec Z$ and some setting $(\vec x',\vec w')$ of the variables in $(\vec X,\vec W)$ such that if $(M,\vec u)\models \vec Z = z^*$ for all $Z\in \vec Z$, then both of the following conditions hold:

              (a) $M \models[\vec X \leftarrow \vec x', \vec W \leftarrow \vec w']
                  PV = \F
                  \wedge \vec V = \vec v
                  \wedge  \vec v \in \mathcal{E}$.

              (b) $M \models[\vec X\leftarrow \vec x, \vec W' \leftarrow \vec w', \vec Z'\leftarrow \vec z^*]
                  \vec V = \vec v
                  \wedge 
                  (\vec v \in \mathcal{E} \Rightarrow PV = \T)
                  $
              for all subsets $\vec W'$ of $\vec W$ and all subsets $Z'$ of $\vec Z$.

        \item  \textbf{AC3.} $\vec X$ is minimal; no subset of $\vec X$ satisfies conditions $AC1$ and $AC2$.
    \end{itemize}
    Where $\vec v$ is the value of endogenous variables.
\end{definition}
\pagebreak

