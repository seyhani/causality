\section{Causal Model of Unsafe Behavior in Event Structure}

In this section, we define a causal model for the safety violation
in an event structure. 
More specifically, we model relations of an event structure 
with the variables and structural equations. 
We then encode the safety violation as the inclusion of a specific
subset of events in the set of configurations of the event structure.


\subsection{Causal Model of Event Structure}
Let $\mathrm{E} = (E,\#,\vdash)$ be an event structure where
$E = \s{e_1, e_2, ...,e_n}$.
Let $S \subseteq \mc{P}(E)$ be the set of unsafe behaviors.
We define the causal model
$\mc{M} = (\mc{S},\mc{F},\mc{E})$ of unsafe behavior
in $\mr{E}$ where
$\mathcal{S} = (\mathcal{U},\mathcal{V},\mathcal{R})$.
We define $\mathcal{U}$ to be empty and $\mathcal{V}$
consisting of boolean variables as follows:
\begin{align*}
    \mathcal{V} = & \s{C_{e_i,e_j} ~|~  1 \leq i < j \leq n.
    e_i \in E \wedge e_j \in E}                                \\
                  & \cup \s{EN_{s,e} ~|~ s \in \mathcal{P}(E),
    e \in E. e \not \in s }                                    \\
                  & \cup \s{M_{s,e} ~|~ s \in \mathcal{P}(E),
        e \in E. e \not \in s }
\end{align*}
Intuitively, these variables model the existence of specific elements in
each of the event structure relations: $\#$, $\vdash_{min}$, and $\vdash$.
For two events $e,e' \in E$, variables of the form $C_{e,e'}$ represent whether $e\#e'$.
Similarly, for a subset of events $s \subseteq E$ and an event $e \in E$,
we use variables of the form $M_{s,e}$ and $EN_{s,e}$ to represent
whether $s \vdash_{min} e$ and $s \vdash e$ respectively.
Finally, we use a single boolean variable, $PV$, which denote the
violation of property in the event structure.
For each variable $X \in \mathcal{V}$ we define $\vec V_X$
as a vector of all variables in $\mathcal{V} \setminus \s{X}$.
For $x,y \in \mathcal{P}(E)$ we say $x$ is covered by $y$ written $ x \prec y$ iff:
\begin{align*}
    x \subseteq y \wedge x \neq y \wedge
    (\forall z. x \subseteq z \subseteq y \Rightarrow x = z
    \vee y = z)
\end{align*}
Next we define the structural equations for each of these variables.
First, we define the structural equation of conflict variables as 
follows:
$$
    \f{C_{e,e'}} = \begin{cases}
        true  & \text{ if } e \# e' \\
        false & \text{ otherwise }
    \end{cases}
$$
We use the existing conflicts in the event structure as the initial 
value for these variables.

\noindent
For minimal enabling variables we define the following equations:
$$
    \f{M_{s,e}} = \begin{cases}
        Min(s,e) \wedge Con(s) & \text{ if } s \vdash_{min} e \\
        false                  & \text{ otherwise }
    \end{cases}
$$
Where we have:
\begin{align*}
    Con(s)   & =   \left(
    \bigwedge_{ 1\leq j<j' \leq n \wedge e_j,e_{j'} \in s}
    \neg C_{e_j,e_{j'}}
    \right)               \\
    Min(s,e) & = \left(
    \bigwedge_{s' \subseteq E. (s' \subset s \vee s \subset s')
        \wedge e \notin s'}
    \neg M_{s',e}
    \right)
\end{align*}
Intuitively, we defined the equation so that it can be affected by 
other conflict and minimal enabling variables.
For instance, assume that for some $s$ we have $s \vdash_{min} e$
thus the variable $M_{s,e}$ is true.
If we add a conflict between any pair of events in $s$ then $M_{s,e}$
becomes false.
Also if make any subset or superset of $s$ to minimally enable the $e$ then
$M_{s,e}$ becomes false.
Finally, we define the equation for the enabling variables as follows:
\begin{align*}
    \f{EN_{s,e}} & =
    \left(
    M_{s,e} \bigvee
    \left(
    \bigvee_{s'\prec s}EN_{s',e}
    \right)
    \right)
    \bigwedge
    Con(s)
\end{align*}
Intuitively, we are capturing the idea that if $s$ enables $e$ then first 
it must be consistent, and secondly either $s$ minimally enables $e$ or 
there exists a subset of $s$ that enables $e$.

\noindent Let $S \subseteq E$ be a subset of events.
We define $\varphi_S$, a boolean formula of primitive
events as follows:
\begin{align*}
    Con(S)
    \bigwedge
    \left(
        \bigwedge_{\forall e \in S}
        \left(
            \bigvee_{\forall \pi \in \pi_S} 
            \left(
                EN_{\e,\pi_1} \wedge
                EN_{\s{\pi_1},\pi_2} \wedge
                \dots
                \wedge
                EN_{\s{\pi_1,...,\pi_{n-1}},\pi_n}
            \right)
        \right)
    \right)
\end{align*}
\noindent Where $\pi_S$ is the set of all permutations of $S$.
For a permutation $\pi \in \pi_S$, where $|S| = n$, 
we write $\pi_i$ where $1 \leq i \leq n$ for the $i$th 
element of $\pi$.
Thus, for a given subset of events $S \subseteq E$, 
we can use $\varphi_S$ as the effect for which we look
for a cause.

\pagebreak

