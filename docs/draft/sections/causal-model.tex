\section{Causal Model of Unsafe Behavior in Event Structure}

In this section, we define a causal model for the safety violation
in an event structure. 
More specifically, we model relations of an event structure 
with the variables and structural equations. 
We then encode the safety violation as the inclusion of a specific
subset of events in the set of configurations of the event structure.


\subsection{Causal Model of Event Structure}
Let $\mathrm{E} = (E,\#,\vdash)$ be an event structure where
$E = \s{e_1, e_2, ...,e_n}$ and $S \in \mc{F}(\mr{E})$ be an unsafe
configuration for which we wish to find the cause.
Let $\mc{M} = (\mc{S},\mc{F},\mc{E})$ be the causal model of the unsafe 
behavior where $\mathcal{S} = (\mathcal{U},\mathcal{V},\mathcal{R})$.
We define $\mathcal{U}$ to be empty and $\mathcal{V}$
consisting of boolean variables as follows:
\begin{align*}
    \mathcal{V} = & \s{C_{e_i,e_j} ~|~  1 \leq i < j \leq n.
    e_i \in E \wedge e_j \in E}                                \\
                  & \cup \s{EN_{s,e} ~|~ s \in \mathcal{P}(E),
    e \in E. e \not \in s }                                    \\
                  & \cup \s{M_{s,e} ~|~ s \in \mathcal{P}(E),
        e \in E. e \not \in s }
\end{align*}
Intuitively, these variables model the existence of specific elements in
each of the event structure relations: $\#$, $\vdash_{min}$, and $\vdash$.
For two events $e,e' \in E$, variables of the form $C_{e,e'}$ represent whether $e\#e'$.
Similarly, for a subset of events $s \subseteq E$ and an event $e \in E$,
we use variables of the form $M_{s,e}$ and $EN_{s,e}$ to represent
whether $s \vdash_{min} e$ and $s \vdash e$ respectively.
For instance, for the event structure of example \ref{example:loop}, 
we have a variable $C_{p,q}$ which denotes a conflict between events 
$p$ and $q$.
We also have variables such $M_{\s{p,q},bb}$ and $EN_{\s{p,q},bb}$ that denote 
whether $\s{p,q} \vdash_{min} bb$ and $\s{p,q} \vdash bb$ respectively.

In the following, for each variable $X \in \mathcal{V}$ we define $\vec V_X$
as a vector of all variables in $\mathcal{V} \setminus \s{X}$.
For $x,y \in \mathcal{P}(E)$ we say $x$ is covered by $y$ written $ x \prec y$ iff:
\begin{align*}
    x \subseteq y \wedge x \neq y \wedge
    (\forall z. x \subseteq z \subseteq y \Rightarrow x = z
    \vee y = z)
\end{align*}
Next we define the structural equations for each of these variables.
First, we define the structural equation of conflict variables as 
follows:
$$
    \f{C_{e,e'}} = \begin{cases}
        true  & \text{ if } e \# e' \\
        false & \text{ otherwise }
    \end{cases}
$$
We use the existing conflicts in the event structure as the initial 
value for these variables.
For the causal model of example \ref{example:loop}, we have 
\begin{align*}
    \f{C_{p,q}} & = false \\
    \f{C_{q,bb}} & = true 
\end{align*}

\noindent
For minimal enabling variables, we define the following equations:
$$
    \f{M_{s,e}} = \begin{cases}
        Min(s,e) \wedge Con(s) & \text{ if } s \vdash_{min} e \\
        false                  & \text{ otherwise }
    \end{cases}
$$
Where we have:
\begin{align*}
    Con(s)   & =   \left(
    \bigwedge_{ 1\leq j<j' \leq n \wedge e_j,e_{j'} \in s}
    \neg C_{e_j,e_{j'}}
    \right)               \\
    Min(s,e) & = \left(
    \bigwedge_{s' \subseteq E. (s' \subset s \vee s \subset s')
        \wedge e \notin s'}
    \neg M_{s',e}
    \right)
\end{align*}
Intuitively, we defined the equation so that it can be affected by 
other conflict and minimal enabling variables.
For instance, assume that for some $s$ we have $s \vdash_{min} e$
thus the variable $M_{s,e}$ is true.
If we add a conflict between any pair of events in $s$ then $M_{s,e}$
becomes false.
Also if make any subset or superset of $s$ to minimally enable the $e$ then
$M_{s,e}$ becomes false.
For the causal model of example \ref{fig:loop}, we have the following 
equation for $M_{\s{p,q},bb}$:
\begin{align*}
    \f{M_{\s{p},bb}} = & Min(\s{p},bb) \wedge Con(\s{p},bb) \\
    Min(\s{p},bb) = & \neg M_{\e,bb} \wedge \neg M_{\s{p,q},bb} \\
    Con(\s{p}) = & true 
\end{align*}
Finally, we define the equation for the enabling variables as follows:
\begin{align*}
    \f{EN_{s,e}} & =
    \left(
    M_{s,e} \bigvee
    \left(
    \bigvee_{s'\prec s}EN_{s',e}
    \right)
    \right)
    \bigwedge
    Con(s)
\end{align*}
Intuitively, we are capturing the idea that if $s$ enables $e$ then first 
it must be consistent, and secondly either $s$ minimally enables $e$ or 
there exists a subset of $s$ that enables $e$.
We have the following equation for $EN_{\s{p},bb}$ of example \ref{fig:loop}:
\begin{align*}
    \f{EN_{\s{p},bb}} = 
    \left(
        M_{\s{p},bb} \vee EN_{\e,bb}
    \right) \wedge Con(\s{p})
\end{align*}

Finally, we need to define the effect for which we seek a cause.
Given that $S \in \mc{F}(\mr{E})$ is an unsafe configuration of $\mr{E}$,
we define $\varphi_S$, a boolean formula of primitive events as follows:
\begin{align*}
    Con(S)
    \bigwedge
    \left(
        \bigwedge_{\forall e \in S}
        \left(
            \bigvee_{\forall \pi \in \pi_{S \setminus e}} 
            \left(
                EN_{\e,\pi_1} \wedge
                EN_{\s{\pi_1},\pi_2} \wedge
                \dots
                \wedge
                EN_{\s{\pi_1,...,\pi_{n-1}},e}
            \right)
        \right)
    \right)
\end{align*}
\noindent Where $\pi_S$ is the set of all permutations of $S$.
For a permutation $\pi \in \pi_S$, where $|S| = n$, 
we write $\pi_i$ where $1 \leq i \leq n$ for the $i$th 
element of $\pi$.
The first and second clause of this formula are defined according to the 
consistent and secured conditions of the configuration in the definition 
\ref{conf}.
Now if we consider the example \ref{example:loop}, and $\s{p,bb}$ as an 
unsafe configuration, we can define $\varphi_{\s{p,bb}}$ as follows:
\begin{align*}
    Con(\s{p,bb}) \wedge
    \left(
        EN_{\e,bb} \wedge EN_{\s{bb},p}
    \right) \wedge
    \left(
        EN_{\e,p} \wedge EN_{\s{p},bb}
    \right)
\end{align*}
Where we have:
\begin{align*}
    Con(\s{p,bb}) = \neg C_{p,bb}
\end{align*}
Using our model we can prove that $M_{\s{q},p} = \F$ is an actual cause 
of $\varphi_{\s{p,bb}}$.

\pagebreak

