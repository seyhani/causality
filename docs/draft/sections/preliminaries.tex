\section{Preliminaries}
\subsection{Event Structure \cite{es}}
\begin{definition}[Event Structure]
    An event structure is a triple $\mr{E} = (E,\#,\vdash)$ where:
    \begin{enumerate}
        \item $E$ is a set of events
        \item \# is a binary symmetric, irreflexive relation on $E$,
              the conflict relation.
              We shall write $Con$ for the set of conflict-free subsets of $E$,
              i.e. those finite subsets $X \subseteq E$ for which:
              $\forall e,e' \in X . \neg (e\#e')$
        \item $\vdash \subseteq Con \times E$ is the enabling relation which satisfies:
              $ X \vdash e \ \& \ X \subseteq Y \in Con \Rightarrow Y \vdash e$
    \end{enumerate}

\end{definition}
\begin{notion}
    In an event structure we shall write $\doublevee$ for the reflexive conflict relation by which we mean
    that $e\doublevee e'$ in an event structure iff either $e\#e'$ or $e=e'$.
    With this notion instead of describing the conflict-free sets of an event structure
    as those sets $X$ such that
    \begin{align*}
        \forall e,e' \in X. \neg(e\#e')
    \end{align*}
    we can say they are those sets $X$ for which:
    \begin{align*}
        \forall e,e' \in X. e\doublevee e' \Rightarrow e=e'
    \end{align*}
\end{notion}

\begin{notion}
    For any event structure we can define the minimal enabling relation $\vdash_{min}$ by:
    \begin{align*}
        X \vdash_{min} e \iff X \vdash e \wedge
        ( \forall Y \subseteq X . Y \vdash e \Rightarrow Y = X )
    \end{align*}
\end{notion}

\begin{definition}[Configuration]
    \label{conf}
    Let $\mr{E} = (E,\#,\vdash)$ be an event structure.
    Define a configuration of $\mr{E}$ to be a subset of events $x \subseteq E$ which is
    \begin{enumerate}
        \item conflict-free: $x \in Con$
        \item secured: $\forall e \in x. \exists e_0,...,e_n \in x. e_n = e \ \wedge
                  \forall i \leq n. \s{e_0,...,e_{i-1}} \vdash e_i$
    \end{enumerate}
\end{definition}
\noindent The set of all configurations of an event structure is written as $\mathcal{F}(E)$.
It is helpful to unwrap condition (2) a little. It says an event $e$ is secured in a set $x$
iff there is a sequence of events $e_0,...,e_n = e$ in $x$ such that:
\begin{align*}
    \emptyset \vdash e_0, \s{e_0} \vdash e_1, ..., \s{e_0,...,e_{i-1}} \vdash e_i,...,
    \s{e_0,...,e_{n-1}} \vdash e_n.
\end{align*}
We call such a sequence $e_0,e_1,...,e_n = e$ a \emph{securing} for $e$ in $x$.

\begin{definition}[Sub-Structure]
    Let $\mr{E_0} = (E_0,\#_0,\vdash_0)$ and $\mr{E_1} = (E_1,\#_1,\vdash_1)$
    be event Structures. Define
    \begin{align*}
        \mr{E_0} \trianglelefteq \mr{E_1} \iff
         & E_0 \subseteq E_1,                                                  \\
         & \forall e,e' \in E_0. e\#_0e'  \iff e,e' \in E_1
        \ \wedge \ e\#_1 e' \text{ and }                                       \\
         & \forall X \subseteq E_0,e \in E_0.X\vdash_0 e  \iff X \subseteq E_0
        \ \wedge \ e \in E_0\ \wedge \ X \vdash_1 e
    \end{align*}
    In this case say $\mr{E_0}$ is a substructure of $\mr{E_1}$.
\end{definition}

\begin{definition}[Restriction]
    Let $\mr{E} = (E,\#,\vdash)$ be an event structure.
    Let $A \subseteq E$.
    Define the restriction of $\mr{E}$ to $A$ to be
    \begin{align*}
        \mr{E} \lceil A = (A,\#_A,\vdash_A)
    \end{align*}
    where
    \begin{align*}
        X \in Con_A \iff X \subseteq A \ \wedge \ X \in Con \\
        X \vdash_A e \iff X \subseteq A \ \wedge \ e \in A \ \& \ X \vdash e
    \end{align*}
\end{definition}

\begin{definition}[Prefixing]
    Let $a$ be an event.
    For an event structure $\mr{E} = (E,\#,\vdash)$ define $a\mr{E}$ to be the event structure $(E',\#',\vdash')$ where:
    \begin{align*}
         & E' = \s{(0,a)} \cup \s{(1,e)|e \in E},                                                                                                        \\
         & \forall e_0',e_1'\in E' . e_0' \#' e_1'  \iff \exists e_0,e_1.e_0' = (1,e_0)
        \ \wedge \ e_1' = (1,e_1) \ \wedge \ e_0 \# e_1                                                                                                  \\
         & \forall X \subseteq E' . X \vdash' e' \iff e' = (0,a) \text{ or } [e' = (1,e_1) \ \wedge \ (0,a)\in X \ \wedge \ \s{e|(1,e)\in X} \vdash e_1]
    \end{align*}
\end{definition}


\begin{definition}[Labelled Event Strtucture]
    A labelled event structure consists of $(E,\#,\vdash,L,l)$ where
    $(E,\#,\vdash)$ is an event structure, $L$ is a set of labels,
    not including the element *, and $l$ is a function $l: E \rightarrow L$
    from its events to its labels.
\end{definition}
\subsection{Constructions on Labeled Event Structures \cite{es}}
We use $(\e,\e)$, an empty set of events and an empty set of labels, to
denote an empty labeled event structure.
In the following, we define four types of constructions on labeled
event structures.

\subsubsection{Prefix}
\begin{definition}[Prefix]
    Let $(E,\#,\vdash,L,l)$ be a labelled event structure.
    Let $\alpha$ be a label.
    Define $\alpha(E,\#,\vdash,L,l)$ to be a labelled event structure
    $(E',\#',\vdash',L',l')$ where $(E',\#',\vdash') = \alpha(E,\#,\vdash)$
    and labels are defined as follows:
    \begin{align*}
        L' = \s{\alpha} \cup L
    \end{align*}
    and
    $$
        l'(e') = \begin{cases}
            \alpha & \text{if } e' = (0,\alpha) \\
            l(e)   & \text{if } e' = (1,e)
        \end{cases}
    $$
    for all $e' \in E'$.
\end{definition}

\subsubsection{Sum}
\begin{definition}[Sum]
    Let $\mr{E_0} = (E_0,\#_0,\vdash_0,L_0,l_0)$ and
    $\mr{E_1} = (E_1,\#_1,\vdash_1,L_1,l_1)$ be labelled event structures.
    Their sum $\mr{E_0} + \mr{E_1}$, is defined to be the structure $(E,\#,\vdash,l)$
    with events $E = \s{(0,e)|e \in E_0} \cup \s{(1,e)|e \in E_1}$,
    the disjoint union of sets $E_0$ and $E_1$,
    with injections $\iota_k: E_k \rightarrow E$, given by
    $\iota_k(e) = (k,e)$, for $k=0,1$, conflict relation
    \begin{align*}
        \forall e,e' \in E.
        e \# e' \iff & \exists e_0,e_0' \in E_0. e = \iota_0(e_0)
        \wedge e' = \iota_0(e_0') \wedge e_0 \#_0e_0'                                                      \\
                     & \text{or } \exists e_1,e_1' \in E_1. e = \iota_1(e_1) \wedge
        e' = \iota_1(e_1') \wedge e_1 \#_1 e_1'                                                            \\
                     & \text{or } \exists e_0 \in E_0,e_1 \in E_1.(e=\iota_1(e_0) \wedge e' =\iota_1(e_1)) \\
                     & \text{or } (e'=\iota_1(e_0) \wedge e =\iota_1(e_1))
    \end{align*}
    and enabling relation
    \begin{align*}
        X \vdash e \iff & X \in Con \wedge e \in E \wedge                   & \\
                        & (\exists X_0 \in Con_0,e_0 \in E_0.X = \iota_0X_0
        \wedge e = \iota_0(e_0) \wedge X_0 \vdash_0 e_0) \text{ or }          \\
                        & (\exists X_1 \in Con_1,e_1 \in E_1.X = \iota_1X_1
        \wedge e = \iota_1(e_1) \wedge X_1 \vdash_1 e_1)                      \\
    \end{align*}
    Where $Con_0,Con_1$ are defined on $\mr{E}_0$ and $\mr{E}_1$ respectively.
    We define the set of labels as $L_0 \cup L_1$ and the labelling function as:
    $$
        l(e) = \begin{cases}
            l_0(e_0) & \text{ if } e = \iota_0(e_0) \\
            l_1(e_1) & \text{ if } e = \iota_1(e_1)
        \end{cases}
    $$
\end{definition}

\subsubsection{Product}
In the product of two event structures, their events of synchronization are those pairs of events $(e_0,e_1)$, one from each event structure;
if $e_0$ is labelled $\alpha_0$ and $e_1$ is labelled $\alpha_1$ the synchronization event is
then labelled $(\alpha_0,\alpha_1)$.
Events need not synchronize however; an event in one component may not synchronize with
any event in the other.
We shall use events of the form $(e_0,*)$/$(*,e_1)$ to stand for the occurrence of an event $e_0$/$e_1$
from one component unsynchronized with any event of the other.
Such an event will be labeled by $(\alpha_0,*)$ where $\alpha_0$ is the original label of $e_0$
and * is a sort of undefined.

\begin{notion}
    In an event structure, we shall write $\doublevee$ for the reflexive conflict
    relation by which we mean that $E \doublevee e'$ in event structure
    iff either $e\#e'$ or $e=e'$.
    With this notation instead of describing the conflict-free sets of an
    event structure as those sets $X$ such that
    \begin{align*}
        \forall e,e' \in X. \neg(e \# e')
    \end{align*}
    we can say they are those sets $X$ for which
    \begin{align*}
        \forall e,e' \in X . e \doublevee e' \Rightarrow e = e'
    \end{align*}
\end{notion}

\begin{definition}[Product]
    Let $\mr{E_0} = (E_0,\#_0,\vdash_0,L_0,l_0)$ and
    $\mr{E_1} = (E_1,\#_1,\vdash_1,L_1,l_1)$
    be labeled event structures.
    Define their product $\mr{E_0} \times \mr{E_1}$ to be the structure $\mr{E} = (E,\#,\vdash,L,l)$
    consisting of events $E$ of the form
    \begin{align*}
        E_0 \times_* E_1 =
        \s{(e_0,*)|e_0 \in E_0}
        \cup \s{(*,e_1)|e_1 \in E_1}
        \cup \s{(e_0,e_1)| e_0 \in E_0 \wedge e_1 \in E_1}
    \end{align*}
    with projections $\pi_i : E \rightarrow_* E_i$,
    given by $\pi_i(e_0,e_1) = e_i$, for $i=0,1$, reflexive conflict relation $\doublevee$ given by
    \begin{align*}
        e \doublevee e' \iff \pi_0(e) \doublevee_0 \pi_0(e') \text{ or }
        \pi_1(e) \doublevee_1 \pi_1(e')
    \end{align*}
    for all $e,e' \in E$ we use $Con$ for the conflict-free finite sets,
    enabling relation $\vdash$ given by
    \begin{align*}
         & X \vdash e \iff X \in Con \wedge e \in E \wedge                  \\
         & (\pi_0(e)\text{ is defined } \Rightarrow \pi_0X\vdash_0\pi_0(e))
        \wedge (\pi_1(e)\text{ is defined } \Rightarrow \pi_1X\vdash_1\pi_1(e))
    \end{align*}
    Its set of labels is
    \begin{align*}
        L_0 \times_* L_1 = \s{ (\alpha_0,*)|\alpha_0 \in L_0}
        \cup \s{(*,\alpha_1)|\alpha_1 \in L_1}
        \cup \s{(\alpha_0,\alpha_1)|\alpha_0 \in L_0 \wedge \alpha_1 \in L_1}
    \end{align*}
    with projections: $\lambda_i: E \rightarrow_* E_i$ given by
    $\lambda_i(\alpha_0,\alpha_1) = \alpha_i$, for $i=0,1$.
    Its labeling function is defined to act on an event $e$ so
    \begin{align*}
        l(e) = (l_0\pi_0(e),l_1\pi_1(e))
    \end{align*}
\end{definition}

\subsubsection{Restriction}

\begin{definition}[Restriction]
    Let $\mr{E} = (E,\#,\vdash,L,l)$ be a labelled event structure.
    Let $\Lambda$ be a subset of labels.
    Define the restriction $\mr{E}\lceil \Lambda$ to be $(E',\#',\vdash',L\cap \Lambda,l')$
    where $(E',\#',\vdash')$ is the restriction of $(E,\#,\vdash)$
    to events $\s{e \in E|l(e) \in \Lambda}$ and the labeling function $l'$
    is the restriction of the original labeling function to the domain $L \cap \Lambda$.
\end{definition}

\subsection{Causal Model \cite{hp}}
A signature $\mathcal{S}$ is a tuple $(\mathcal{U},\mathcal{V},\mathcal{R})$,
where $\mathcal{U}$ is a set of exogenous variables, $\mathcal{V}$
is a set of endogenous variables, and $R$ associates with every variable
$Y\in \mathcal{U}\cup \mathcal{V}$ a nonempty set $\mathcal{R}(Y)$ of possible values for $Y$.
A causal model (or structural model) over signature $S$ is a tuple
$M=(\mathcal{S},\mathcal{F})$, where $\mathcal{F}$ associates with
each variable $X \in \mathcal{V}$ a function denoted $F_X$ such that
$F_X: (\times_{U\in \mathcal{U}}\mathcal{R}(U))\times (\times_{Y\in\mathcal{V}-\{X\}}\mathcal{R}(Y))\rightarrow \mathcal{R}(X)$.

$F_X$ determines the value of $X$ given the values of all the other variables
in $\mathcal{U}\cup \mathcal{V}$.
For example, if $F_X(Y,Z,U)=Y+U$ (which we usually write as $X = Y + U$),
then if $Y=3$ and $U=2$, then $X = 5$, regardless of how $Z$ is set.
These equations can be thought of as representing processes (or mechanisms) by which values are assigned to variables. Hence, like physical laws, they support a counterfactual interpretation.
For example, the equation above claims that in the context $U=u$, if $Y$ were 4, then $X$ would be $u+4$ (which we write as $(M,u) \models [Y\leftarrow 4](X = u + 4))$, regardless of what values X, Y, and Z actually take in the real world.


The function $\mathcal{F}$ defines a set of (\textit{modifiable}) \textit{structural equations} relating to the values of the variables.

\begin{example}
    Suppose that we want to reason about a forest fire that could
    be caused by either lightning or a match lit by an arsonist.
    Then the causal model would have the following endogenous variables :
    \begin{itemize}
        \item $F$ for fire
        \item $L$ for lighting
        \item $ML$ for match lit
    \end{itemize}
    The set $\mathcal{U}$ of exogenous variables includes conditions
    that suffice to render all relationships deterministic (such as
    whether the wood is dry, whether there is enough oxygen in the air
    for the match to light, etc.).
    Suppose that $\vec u$ is a setting of the exogenous variables that
    makes a forest fire possible (i.e., the wood is sufficiently dry,
    there is oxygen in the air, and so on).
    Then, for example, $F_F(\vec u, L, ML)$ is such that $F=1$ if either
    $L=1$ or $ML=1$.
    Note that although the value of $F$ depends on the value $L$ and $ML$, the value of $L$ does not depend on the values of $F$ and $ML$.
\end{example}

\subsubsection{Causal Network}
We can describe a causal model $\m$ using a causal network.
This is a graph with nodes corresponding to the random variables
in $\mathcal{V}$ and an edge from a node labeled $X$ to one
labeled $Y$ if $F_Y$ depends on the value of $X$.
This graph is a dag that follows from the assumption that the
equations are recursive.
We occasionally omit the exogenous variables $\vec U$ from the causal network.
For example, the causal network for example 2.2.1 has the following
form:

\begin{center}
    \begin{tikzpicture}[node distance={15mm}]
        \node (l) {L};
        \node (ml) [below right of=l]  {ML};
        \node (f) [above right of=ml] {F};
        \draw [->] (l) -- (ml);
        \draw [->] (f) -- (ml);
    \end{tikzpicture}
\end{center}

\subsubsection{Actual Cause}
Given a signature $S= (\mathcal{U},\mathcal{V},\mathcal{R})$, a formula of the form $X =x$, for $X \in \mathcal{V}$ and $x \in \mathcal{R}(X)$, is called a \textit{primitive event}.
A \textit{basic causal formula} is one of the form $[Y_1 \leftarrow y_, ..., Y_l\leftarrow y_k]\varphi$, where $\varphi$ is a Boolean combination of primitive events, $Y_1,...,Y_k$ are distinct variables in $\mathcal{V}$, and $y_i \in \mathcal{R}(Y_i)$.
Such a formula is abbreviated as $[\vec{Y}\leftarrow\vec{y}]\varphi$.
A \textit{causal formula} is a Boolean combination of basic causal formulas.
A causal formula $\psi$ is true or false in a causal model, given a context.
We write $(M,\vec u)\models \psi$ if $\psi$ is true in causal model $M$ given context $\vec u$.
$(M,\vec u)\models [\vec Y\leftarrow \vec y](X=x)$ if the variable $X$ has value $x$ in the unique solution to the equation in $M_{\vec{Y} \leftarrow \vec{y}}$ in context $\vec u$.
The context and structural equations are given.
They encode the background knowledge.
All relevant events are known.
The only question is picking out which of them are the cause of $\varphi$ or, alternatively, testing whether a given set of events can be considered the cause of $\varphi$.
The types of events that we allow as actual causes are ones of the form $X_1 = x_1 \wedge ... \wedge X_k=x_k$-- that is, conjunctions of primitives events.
We abbreviate this as $\vec X = \vec x$.
\begin{definition}[Actual Cause]
    $\vec X = \vec x$ is an actual cause of $\varphi$ in $(M,\vec u)$ if the following three conditions hold:
    \begin{itemize}
        \item  \textbf{AC1.} $(M,\vec u)\models (\vec X = \vec x) \wedge \varphi$.
              (both $\vec X = \vec x$ and $\varphi$ are true in actual world)
        \item  \textbf{AC2. }There exists a partition $(\vec Z, \vec W)$ of $\mathcal{V}$ with $\vec X \subseteq \vec Z$ and some setting $(\vec x',\vec w')$ of the variables in $(\vec X,\vec W)$ such that if $(M,\vec u)\models \vec Z = z^*$ for all $Z\in \vec Z$, then both of the following conditions hold:

              (a) $(M,\vec u)\models[\vec X \leftarrow \vec x', \vec W \leftarrow \vec w']\neg \varphi$.

              (b) $(M,\vec u)\models[\vec X\leftarrow \vec x, \vec W \leftarrow \vec w', \vec Z'\leftarrow \vec z^*]\varphi$ for all subsets $Z'$ of $\vec Z$.

        \item  \textbf{AC3.} $\vec X$ is minimal; no subset of $\vec X$ satisfies conditions $AC1$ and $AC2$.
    \end{itemize}
\end{definition}
We call the tuple $(\vec W, \vec w,\vec x')$ a witness to the fact that $\vec X=\vec x$ is a cause of $\varphi$.

\begin{definition}[But-For Cause]
    We say $\vec X = \vec x$ is a but-for cause of $\varphi$ in
    $(M,\vec u)$ if there exists a witness $(\vec W, \vec w, \vec x')$
    for $\vec X = \vec x$ being an actual cause of $\varphi$
    where $\vec W = \emptyset $.
\end{definition}

Note that, if we consider a witness $(\vec W, \vec w, \vec x')$
for checking whether $\vec X = \vec x$ is a cause of $\varphi$
in $(M,\vec u)$ where $\vec W = \e$, then in the AC2(b) condition
we only need to check whether $(M,\vec u) \vDash [\vec X \leftarrow \vec x, \vec Z' \leftarrow \vec z^*]\varphi$ for all subsets $\vec Z'$
of $\vec Z$.
Since we have $(M,\vec u) \vDash (\vec X = \vec x)$ and
$(M,\vec u) \vDash Z = z^*$ for all $Z \in \vec Z$,
the Interventions $\vec X \leftarrow \vec x$ and
$\vec Z ' \leftarrow \vec z^*$ actually do not change the value of
any variable thus checking whether
$(M,\vec u) \vDash [\vec X \leftarrow \vec x, \vec Z' \leftarrow \vec z^*]\varphi$ is true
reduces to check whether $(M,\vec u) \vDash \varphi$
which must be already satisfied when we have checked AC1 condition.
This means that, to check whether $\vec X = \vec x$ is an actual cuase when using a witness with an empty $\vec W$
we only need to check AC1 and AC2(a) conditions.
\subsection{Extended Causal Model}
An extended causal model is a tuple $(\mathcal{S},\mathcal{F},
    \mathcal{E})$, where $(\mathcal{S},\mathcal{F})$ is a causal model, and $\mathcal{E}$ is a set of allowable settings for the endogenous variables.
That is, if the endogenous variables are $X_1,...,X_n$ then
$(x_1,...,x_n) \in \mathcal{E}$ if $X_1 = x_1, ..., X_n=x_n$ is an
allowable setting.
We say that a setting of a subset of the endogenous variables is allowable if it can be extended to a setting in $\mathcal{E}$.

In \cite{hp} there is no formal definition of the actual cause
in extended causal models.
So, here we provide a new definition of the actual cause
in the context of extended causal models.


\begin{definition}
    $\vec X = \vec x$ is an actual cause of $\varphi$ in $(M,\vec u)$ if the following three conditions hold:
    \begin{itemize}
        \item  \textbf{AC1.} $(M,\vec u)\models (\vec X = \vec x) \wedge \varphi \wedge $.
        \item  \textbf{AC2. }There exists a partition $(\vec Z, \vec W)$ of $\mathcal{V}$ with $\vec X \subseteq \vec Z$ and some setting $(\vec x',\vec w')$ of the variables in $(\vec X,\vec W)$ such that if $(M,\vec u)\models \vec Z = z^*$ for all $Z\in \vec Z$, then both of the following conditions hold:

              (a) $(M,\vec u)\models[\vec X \leftarrow \vec x', \vec W \leftarrow \vec w']\neg \varphi
                  \wedge \vec V = \vec v
                  \wedge  \vec v \in \mathcal{E}$.

              (b) $(M,\vec u)\models[\vec X\leftarrow \vec x, \vec W'
                      \leftarrow \vec w', \vec Z'\leftarrow \vec z^*]
                  \vec V = \vec v \wedge (\vec v \in \mc{E} \Rightarrow \varphi)$
              for all subsets $Z'$ of $\vec Z$.

        \item  \textbf{AC3.} $\vec X$ is minimal; no subset of $\vec X$ satisfies conditions $AC1$ and $AC2$.
    \end{itemize}
    Where $\vec v$ is the value of endogenous variables.
\end{definition}
