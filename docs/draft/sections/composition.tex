\section{Composition of Actual Cause}

\begin{notion}
    Let $\mc{A}$ be a universal set of events and $\es = (E,\#,\vdash)$
    be an event structure where $E \subseteq \mc{A}$.
    For a subset $C \subseteq \mc{A}$ we write $\es \vDash C$ iff
    $C \in \mc{F}(\es)$.
\end{notion}

\begin{notion}
    Let $\es = (E,\#,\vdash)$ be an event structure and
    $\mc{M}$ be the causal model of $\es \vDash C$  for $C \in \mc{A}$.
    We use $(X,\vec W, \vec w) \hookrightarrow_{\m} \es \vDash C$ to denote
    that $(X,\vec W,\vec w)$ where $X$ and $\vec W$ are variables of
    $\mc{M}$ is a witness of $X$ being an actual cause of
    $\es \vDash C$ in $\mc{M}$.
\end{notion}

\begin{definition}
    Let $\mr{E}_0 = (E_0,\#_0,\vdash_0)$ and
    $\mr{E}_1 = (E_1,\#_1,\vdash_1)$ be event structures
    such that $E_0 \cup E_1 = \emptyset$.
    We redefine the sum operation under this condition.
    Let $ \mr{E} = \mr{E}_0 + \mr{E}_1$ where
    $\mr{E} = (E,\#,\vdash)$.
    We define $E = E_0 \cup E_1$ and the conflict relation as follows:
    \begin{align*}
        \forall e,e' \in E. e \# e' & \iff
        \exists e_0,e_0' \in E_0. e_0 \#_0 e_0'
        \vee \exists e_1,e_1' \in E_1. e_1 \#_1 e_1'                                 \\
                                    & \vee \left(e \in E_0 \wedge e' \in E_1 \right)
        \vee \left(e \in E_1 \wedge e' \in E_0 \right)
    \end{align*}
    And finally we define the enabling relation as follows:
    \begin{align*}
        \forall X \in Con, e \in E. X \vdash e \iff
        \left( X \in Con_0 \wedge X \vdash_0 e \right)
        \vee \left( X \in Con_1 \wedge X \vdash_1 e \right)
    \end{align*}
\end{definition}

\begin{definition}
    Let $\mc{A} = E_0 \cup E_1$ be a set of events such that
    $E_0 \cap E_1 = \e$ and $\mr{E}_0 = (E_0,\#_0,\vdash_0)$,
    $\mr{E}_1 = (E_1,\#_1,\vdash_1)$ be event structures.
    Let $C \subseteq \mc{A}$ be a subset of events.
    Let $\m_0,\m_1$ be causal models of $\es_0 \vDash C,\es_1 \vDash C$
    respectively and let $\mc{V}_0,\mc{V}_1$ be the set of endogenous
    variables of these models.
    Let $\es = \es_0 + \es_1$ and $\m$ be the causal model of
    $\es \vDash C$ with the set of endogenous variables $\mc{V}$.
    Note that we have $\mc{V}_0 \cup \mc{V}_1 \subset \mc{V}$.
    We define the structured causal model $\m^+$ by making all variables
    in $\mc{V}\setminus (\mc{V}_0 \cup \mc{V}_1)$ exogenous and
    defining the context $\vec u^+$ with values of these variables in
    $\m$.
\end{definition}

\begin{theorem}
    Let $\es_0$ and $\es_1$ be event structures and
    $\es = (E,\#,\vdash)$ be their sum.
    Let $C \subseteq E$ be a configuration of $\es_0$
    and $\m_0$, $\m_1$, and $\m$ be the causal models
    of $\es_0 \vDash C$,$\es_1 \vDash C$, and $\es \vDash C$ respectively.
    We have:
    \begin{align*}
        (X,\vec W,\vec w) \hookrightarrow_{\m_0} \es_0 \vDash C
        \Rightarrow (X,\vec W,\vec w) \hookrightarrow_{\m^+} \es \vDash C
    \end{align*}
    The theorem intuitively states that a witness of the actual cause
    of $\es_0 \vDash C$ is also a witness of the actual cause of
    $\es \vDash C$ in $\m^+$.
\end{theorem}

To prove this theorem first we define some new notions and prove some lemmas.

\begin{notion}
    Let $\m$ be a causal model with the set of endogenous variables $\mc{V}$.
    Let $X,Y \in \mc{V}$ be two endogenous variables of $\m$.
    Given a context $\vec u$, we say that $Y$ depends on $X$ denoted by
    $X \to Y$ if there exists a set of variables $\vec W$ and values
    $\vec w, x', y'$ where $x \neq x'$ and $y \neq y'$ that satisfy the
    following conditions:
    \begin{align*}
        (\m,\vec u) \vDash[\vec W \la \vec w] X = x \wedge Y = y \\
        (\m,\vec u) \vDash[\vec W \la \vec w, X \la x'] Y = y'
    \end{align*}
    The dependency intuitively captures the fact that the value of $Y$
    can be changed by changing the value of $X$.
\end{notion}

\begin{notion}
    Let $\m$ be a causal model with the set of endogenous variables $\mc{V}$.
    For a variable $X \in \mc{V}$, we use $\mc{D}(X)$ to denote the set of
    variables that depend on $X$:
    \begin{equation*}
        \mc{D}(X) = \s{Y \in \mc{V} | X \to Y }
    \end{equation*}
\end{notion}
Now assume that there exists a witness $(X,\vec W)$ in $\m_0$ for
which we have $(X,\vec W) \hookrightarrow \mr{E}_0 \vDash C$.
Note that $\mc{V}_0 \subset \mc{V}$, thus all the variables $X$ and
$\vec W$ exist in $mc{V}$.
First, note that $\mr{E}_0 \vDash C$ means that $C \subseteq E$.
$(X,\vec W)$ satisfying the conditions of actual means that
In a causal model $\m$ with the set of endogenous variables
$\mc{V}$, for a variable $X \in \mc{V}$ we use $D(X) \subseteq \mc{V}$
to denote the set of variables in $\mc{V}$ that have $X$ in their function.
Let $\vec u$ be the settings of exogenous variables in $\m$.

\begin{lemma}
    Let $\es_0 = (E_0,\#_0,\vdash_0)$ and $\es_1 = (E_1,\#_1,\vdash_1)$
    be event structures with sum $\es = \es_0 + \es_1$.
    Let $\m_0,\m_1,\m$ be their causal models for an arbitrary
    configuration with the set of endogenous variables $\V_0,\V_1,\V$.
    Let $\m^+$ be the structured causal model of $\m$ with the context
    $\vec u^+$ and set of endogenous variables $\V^+$.
    For each variable $X \in \V^+$, if $X \in \mc{V}_0$
    then the function of $X$ in $(\m^+,\vec u^+)$ equals to its function in $\m_0$.
\end{lemma}

\begin{proof}
    Let $X \in \V^+$ be a variable for which we have $X \in \V_0$ with the
    function $F_0$ in $\m_0$ and $F^+$ in $(\m^+,\vec u^+)$.
    We consider $X$ regarding different type of variables.
    First, let $X = C_{e,e'}$.
    We have $e\#_0e' \iff e\#e'$, thus we can simply
    conclude that $F^+ = F_0$ since the function returns a constant value.
    Next let $X = M_{s,e}$.
    We have $s \vdash_{min}^0 e \iff s \vdash_{min} e$.
    Thus, in case $s \not \vdash_{min}^0 e$, $F^+$ and
    $F_0$ return the same false value.
    Now, assume that $s \vdash_{min} e$.
    We have $F^+ = Min^+(s,e) \wedge Con^+(s)$.
    The only difference in functions is that $Min^+(s,e)$
    depends on the variables $M_{s',e}$ where
    $s'$ is a superset of $s$.
    As a result, these supersets are different in $\m$ and $\m^+$.
    Let $S^+$ and $S_0$ be the set of all supersets of $s$ in $\m^+$ and $\m_0$
    respectively.
    For each $s' \in S^+$ that is not in $S_0$, $M_{s',e}$ is an exogenous
    variable with the value of false.
    Hence, in such a context, by omitting all such variables from
    $Min^+(s,e)$ we essentially achieve the same function.
    Thus, we have $F^+ = F_0$.
    Finally, let $X = EN_{s,e}$.
    The function of $EN$ only includes variables corresponding to events in
    $s$, so they already exist in $\V_0$ and as a result they have equal
    functions in $\m_0$ and $\m^+$.
\end{proof}

\begin{lemma}
    Let $\es_0 = (E_0,\#_0,\vdash_0)$ and $\es_1 = (E_1,\#_1,\vdash_1)$
    be event structures with sum $\es = \es_0 + \es_1$.
    Let $\m_0,\m_1,\m$ be their causal models for an arbitrary
    configuration with the set of endogenous variables $\V_0,\V_1,\V$.
    Let $\m^+$ be the structured causal model of $\m$ with the context
    $\vec u^+$ and set of endogenous variables $\V^+$.
    For a variable $X$ let $D^+(X)$ denotes the of its dependent variables
    in $(\m^+,\vec u^+)$.
    For each variable $X \in \V^+$, if $X \in \mc{V}_0$
    we have $\mc{D}(X) = \mc{D}^+(X)$.
\end{lemma}
\begin{proof}
    For a given variable $X \in \mc{V}_0$, let $\mc{D}_0(X),\mc{D}^+(X)$
    be the set of dependent variables to $X$ in $\m_0$ and $(\m^+,\vec u^+)$
    respectively.
    Let $X = C_{e,e'}$.
    In $\m^+$, all variables of the form $M_{s,e''}$ and $EN_{s,e''}$ has
    $X$ in their function if $e,e' \in s$.
    IF $s \cup e'' \subseteq E_0$ so these variables are included in
    $\mc{D}_0$.
    Otherwise, we have $s\cup e'' \not \subseteq E_0$ and
    $s \cup e'' \not \subseteq E_1$.
    Thus these variables are exogenous and can not be included in $\mc{D}^+$.
    Next, let $X = M_{s,e}$.
    The variable $EN_{s,e}$ and variables $M_{s',e}$ where
    $s' \subset s$ or $s \subset s'$ include $X$ in their function.
    $EN_{s,e}$ and all variables of the form $M_{s',e}$ where
    $s' \subset s$ are exist in $\mc{D}_0(X)$.
    For variables where $s \subset s'$ then if $s' \subseteq E_0$ then
    it exists in $\mc{D}_0(X)$.
    Otherwise,
    If $s' \not \subseteq E_0$ then it is an exogenous variable thus it does
    not belong to $\mc{D}^+(X)$ either.
    Finally, let $X = EN_{s,e}$.
    Variables of the form $EN_{s',e}$ where $s \prec s'$ have $EN_{s',e}$
    in their function.
    If $s' \subseteq E_0$ then $EN_{s',e}$ exists in $\mc{D}_0(X)$
    otherwise, it is an exogenous variable.
    This completes the proof that for each variable $X$ we have
    $\mc{D}^+(X) = \mc{D}(X)$.
\end{proof}

\begin{lemma}
    For all $\vec W, \vec V \subseteq \mc{V}_0$ and a vector of values $\vec v$
    for $\vec V$ we have:
    \begin{align*}
        \m_0 \vDash [\vec W \la \vec w]\vec V = \vec v
        \Rightarrow (\m,\vec u) \vDash [\vec W \la \vec w] \vec V = \vec v
    \end{align*}
    Assume that we set $\vec W \la \vec w$ in $\m$.
    Since for each variable $X \in \mc{V}_0$ we have
    $\mc{D}(X) \subseteq \mc{V}_0$, thus using the lemma 3 the new setting
    only affects variables in $\mc{V}_0$.
    Using the lemma 2 we can conclude that the new values of each affected
    variable has the same value as in $\m_0$.
\end{lemma}

Now we can purpose the proof of the theorem 4.1.
\begin{proof}
    Let $\es_0 = (E_0,\#_0,\vdash_0)$ and $\es_1 = (E_1,\#_1,\vdash_1)$ be event
    structures and $\es = (E,\#,\vdash)$ be their sum.
    Let $C \subseteq E$ be a configuration of $\es_0$
    and $\m_0$, $\m_1$, and $\m$ be the causal models
    of $\es_0 \vDash C$,$\es_1 \vDash C$, and $\es \vDash C$ respectively.
    Let $(X,\vec W) \hookrightarrow_{\m_0} \es_0 \vDash C$.
    This means that $(X,\vec W)$ satisfies the AC conditions.
    To satisfy the AC conditions, we check whether effect holds or not
    under specific intervention.
    To be more precise, we check formulas of the form:
    \begin{equation*}
        \m_0\vDash[\vec R \la \vec r] \psi
    \end{equation*}
    where $\psi$ is either $\es_0 \vDash C$ or $\es_0 \not \vDash C$.
    Since all possible set of variables $\vec R$ we have
    $\vec R \subseteq \mc{V}_0$, thus using the lemma 4 we have:
    \begin{equation*}
        \m_0\vDash[\vec R \la \vec r] \vec V = \vec v
        \Rightarrow
        (\m^+,\vec u^+)\vDash[\vec R \la \vec r] \vec V = \vec v
    \end{equation*}
    For each $\vec V \subseteq \V_0$.
    This means that if the value of the variables in $\V_0$ be $\vec v$
    after making the intervention $\vec R \la \vec r$ in $\m_0$,
    then doing the same intervention in $\m^+$ causes these variables to
    have the same value as they had in $\m_0$.
    Moreover, note that determining whether $C$ is a configuration of
    $\es$ or not only depends on the variables in $\mc{V}_0$.
    So, since the values of these variables has same value after
    the intervention in $\m_0$, thus we can conclude the
    the $\psi$ formulas also holds in $\m$ so, the AC conditions
    are satisfied in $\m^+$ too.
\end{proof}