\documentclass{beamer}

\usepackage{tikz}
\usetikzlibrary{graphs, decorations.pathmorphing}
\usepackage{cancel}

\begin{document}

\begin{frame}{Improved Lattice}
We add new \textit{reason} vertices to the configurations lattice:

\begin{center}
\begin{tikzpicture}
  \node (X) at (0, 1) {$X$};
  \node[green] (s-e) at (-1.5, 0) {$s \vdash e$};
  \node[circle,draw] (reason) at (0, 0) {};
  \node[circle,draw] (reason-X) at (0, -2) {};
  \node[green] (X-f) at (0, -3) {$X_f$};
  \node (X') at (2, -1) {$X'$};
  \node[circle,draw] (reason-X') at (1, -2) {};

  \draw[->] (X) -- (reason);
  \draw[->] (s-e) -- (reason);
  \draw[->,decorate,decoration=snake] (reason) -- (reason-X);
  \draw[->] (reason-X) -- (X-f);
  \draw[->] (X') -- (reason-X');
  \draw[->] (reason-X') -- (X-f);
\end{tikzpicture}
\end{center}

\begin{itemize}
  \item Cause: $s \vdash e$
  \item Effect: At least one configuration from a given set of configurations
  (including $X_f$) is \textbf{valid}.
\end{itemize}
\end{frame}

\begin{frame}{HP: AC1}
Both the cause and the effect hold.

\begin{itemize}
  \item Cause: $s \vdash e$
  \item Effect: $X_f$ is valid.
\end{itemize}
\end{frame}

\begin{frame}{HP: AC2.a}

\begin{center}
\begin{tikzpicture}
  \node (X) at (0, 1) {$X$};
  \node[red] (s-e) at (-1.5, 0) {$s \vdash e$};
  \node[blue] (s'-e') at (1.5, 0) {$s' \vdash e'$};
  \node[circle,draw] (reason) at (0, 0) {};
  \node[circle,draw] (reason-X) at (0, -2) {};
  \node[red] (X-f) at (0, -3) {$X_f$};
  \node[blue] (X') at (2, -1) {$X'$};
  \node[circle,draw] (reason-X') at (1, -2) {};

  \draw[->] (X) -- (reason);
  \draw[->] (s-e) -- (reason);
  \draw[->] (s'-e') -- (reason);
  \draw[->,decorate,decoration=snake] (reason) -- (reason-X);
  \draw[->] (reason-X) -- (X-f);
  \draw[->] (X') -- (reason-X');
  \draw[->] (reason-X') -- (X-f);
\end{tikzpicture}
\end{center}

\begin{itemize}
  \item HP's $x'$: $s \not \vdash e$
  \item HP's Witness: We turn off the following:
  \begin{itemize}
    \item All vertices similar to $s \vdash e$
    \item All configurations similar to $X'$
    \item All other configurations of the effect set
  \end{itemize}
\end{itemize}
\end{frame}

\begin{frame}{HP: AC2.b}

\begin{center}
\begin{tikzpicture}
  \node (X) at (0, 1) {$X$};
  \node[green] (s-e) at (-1.5, 0) {$s \vdash e$};
  \node[blue] (s'-e') at (1.5, 0) {$s' \vdash e'$};
  \node[circle,draw] (reason) at (0, 0) {};
  \node[circle,draw] (reason-X) at (0, -2) {};
  \node[green] (X-f) at (0, -3) {$X_f$};
  \node[blue] (X') at (2, -1) {$X'$};
  \node[circle,draw] (reason-X') at (1, -2) {};

  \draw[->] (X) -- (reason);
  \draw[->] (s-e) -- (reason);
  \draw[->] (s'-e') -- (reason);
  \draw[->,decorate,decoration=snake] (reason) -- (reason-X);
  \draw[->] (reason-X) -- (X-f);
  \draw[->] (X') -- (reason-X');
  \draw[->] (reason-X') -- (X-f);
\end{tikzpicture}
\end{center}

Turning $s \vdash e$ on again will make $X_f$ valid again.
Resetting any other variable (except the witness variables)
will keep $X_f$ valid.
\end{frame}

\begin{frame}{Improved Lattice with Conflicts}
Checking causality with a conflict as a cause
($e \cancel{\#} e_1$):

\begin{center}
\begin{tikzpicture}
  \node (X) at (0, 1) {$X$};
  \node[red] (e-c-e1) at (-1.5, 0) {$e \cancel{\#} e_1$};
  \node (e-c-e2) at (1.5, 0) {$e \cancel{\#} e_2$};
  \node[rectangle,draw] (reason) at (0, 0) {};
  \node[rectangle,draw] (reason-X) at (0, -2) {};
  \node[green] (X-f) at (0, -3) {$X_f$};
  \node[blue] (X') at (3, -1) {$X'$};
  \node[rectangle,draw] (reason-X') at (2, -2) {};

  \node[circle,draw] (reason-old) at (0, -0.7) {};
  \node[circle,draw] (reason-X'-old) at (1, -2.7) {};

  \draw[->] (X) -- (reason);
  \draw[->] (e-c-e1) -- (reason);
  \draw[->] (e-c-e2) -- (reason);
  \draw[->,decorate,decoration=snake] (reason-old) -- (reason-X);
  \draw[->] (reason-X) -- (X-f);
  \draw[->] (X') -- (reason-X');
  \draw[->] (reason) -- (reason-old);
  \draw[->] (reason-X') -- (reason-X'-old);
  \draw[->] (reason-X'-old) -- (X-f);
\end{tikzpicture}
\end{center}

\end{frame}

\begin{frame}{Non-Singleton Causes}
To check AC1, AC2.\{a,b\}, We pick the shortest path from the set of causes to the set
of effects $\rightarrow$ no other cause on the path.

To check AC3, our current solution is to brute-force on all subsets.

\end{frame}

\end{document}