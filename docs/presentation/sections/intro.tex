\section{Overview}

\begin{frame}{Motivation: Example}
    \begin{figure}
        \centering
        \begin{tikzpicture}[node distance={20mm},main/.style = {draw, circle,minimum size=8mm}]
            \node[main] (a)  {$a$};
            \node[main] (b) [above of=a]  {$b$};
            \node[main] (c) [left of=b] {$c$};
            \node[main] (d)  [below of=c] {$d$};
            \draw [->,green,thick] (a) -- (b);
            \draw [->,green,thick] (b) edge[bend right] (c);
            \draw [->,orange,thick] (c) -- (d);
            \draw [->,green,thick,dashed] (a) -- (c);
            \draw [->,orange,thick,dashed] (c) -- (b);
            \draw [->,orange,thick,dashed] (b) -- (d);
            \draw pic["$\alpha$",
            draw=blue,->,thick,angle eccentricity=1.2,
            angle radius=1.2cm] {angle=b--a--c} ;
            \draw pic["$\beta$",
            draw=blue,<-,thick,angle eccentricity=1.2,
            angle radius=0.8cm] {angle=b--c--d} ;
        \end{tikzpicture}
    \end{figure} 
    \begin{itemize}
        \item Two concurrent updates $\alpha$ and $\beta$
        \item Assume that $\beta$ happen followed by the arrival of a packet on $b$
        \item Loop-freedom property is violated
        \item \textbf{Why} the property is violated?
        \item Use the actual cause of the violation to answer to this question
    \end{itemize}
\end{frame}

\begin{frame}{Actual Cause of Property Violation in DyNetKAT}
    \begin{itemize}
        \item DyNetKAT: a software defined network programming language
        \item Event Structure: a non-interleaving computation model of processes
        \item Actual Cause: a mathematical formulation for explaining phenomenons
        \item Goal: reason about the actual cause of property violation 
        in the DyNetKAT programs using event structure as a semantic
        model
    \end{itemize}
\end{frame}

\begin{frame}{Contributions}
    \begin{itemize}
        \item Event structure semantic for Normal DyNetKAT terms
        \item Event structure semantic for unsafe behavior (property violation)
        \item Causal model of unsafe behavior in event structure
        \item Applying the causal reasoning on some categories of network properties
        \item Implementing a prototype of EStResT
    \end{itemize}
\end{frame}
